\documentclass{article}
\title{Anharmonic Group Elements as Generated by Machine}
\author{Ed Rogers}
\date{March 2011}
\begin{document}
   \maketitle

\section{Representing using Creation and Annihilation Operators}
\label{secRepresentationIntro}

The quartic perturbed harmonic oscillator in quantum mechanics can be represented by creation an annihilation operators, like so:

\begin{table}[!hp]
\begin{center}
\begin{tabular}{rcl}
$H_{0} = \frac{(p^2+x^2)}{2}$ & $\rightarrow$ & $H_{0} = BA+\frac{1}{2}$ \\
$H_{4} = H_{0} + \frac{\lambda}{4}{\cdot}(x^{4})$ & $\rightarrow$ & $H_{4} = H_{0} + \frac{\lambda}{4}{\cdot}((B+A)^{4})$ \\
& & \\
$[A,B] = 1$ & & \\
\end{tabular}
\caption{Here, we have defined $B=a^{\dagger}$ and $A=a$, a more handy notation for our needs.\label{tabBuildingBlocks}}
\end{center}
\end{table}

$H_{4}$ can be normal ordered to the following result:

\begin{eqnarray}
H_{4} & = & H_{0} + \frac{\lambda}{4}{\cdot}((B+A)^{4}) \\
      & = & H_{0} + {\lambda}{\cdot}(0.25){\cdot}(B^{4}+A^{4}) + {\lambda}{\cdot}(B^{3}A+BA^{3}) \nonumber \\
      &   &  + {\lambda}{\cdot}(1.5){\cdot}(B^{2}+A^{2}) + {\lambda}{\cdot}(1.5){\cdot}B^{2}A^{2} \nonumber \\
      &   &  + {\lambda}{\cdot}(3){\cdot}BA + {\lambda}{\cdot}(0.75)
\end{eqnarray}

\newpage
\section{Solution to first order in $\lambda$}
\label{secFirstOrder}

\subsection{Step 1: Identify All Elements of the Lie Algebra}

Elements of the Lie Algebra at first order ($L_{m}^{(k)}$ where $k=1$) are determined by performing commutations with $H_{0}$ and $H_{4}$, as identified in Table \ref{tabBuildingBlocks}.  At first order, terms of order $O(\lambda^{2})$ are ignored, so only one commutation is required.

The first commutator:
\begin{table}[!hp]
\begin{center}
\begin{tabular}{rcl}
$[H_{0},H_{4}]$ & = & $[B{\cdot}A,{\lambda}{\cdot}(\frac{A+B}{\sqrt{2}})^{4}]$ \\
 &   &  \\
 & = & ${\lambda}{\cdot}(B^{4}-A^{4}) + {\lambda}{\cdot}(2){\cdot}(B^{3}A-BA^{3}) + {\lambda}{\cdot}(3){\cdot}(B^{2}-A^{2})$ \\
\end{tabular}
\end{center}
\end{table}

At this point, we can identify all the terms of the Lie Algebra to first order.

\begin{table}[!hp]
\begin{center}
\begin{tabular}{rcl}
               &   & \textbf{Nonperturbative Terms} ($\lambda^{k}$ where $k=0$) \\
\hline
$L_{0}^{(1)}$  & = & $I = 1$ \\
$L_{1}^{(1)}$  & = & $BA$ \\
               &   & \textbf{First Order Terms} ($\lambda^{k}$ where $k=1$) \\
\hline         
$L_{2}^{(1)}$  & = & ${\lambda}{\cdot}I = {\lambda}$ \\
$L_{3}^{(1)}$  & = & ${\lambda}{\cdot}BA$ \\
$L_{4}^{(1)}$  & = & ${\lambda}{\cdot}B^{2}A^{2}$ \\
               &   & \\
$L_{5}^{(1)}$  & = & ${\lambda}{\cdot}(B^{4}+A^{4})$ \\
$L_{6}^{(1)}$  & = & ${\lambda}{\cdot}(B^{3}A+BA^{3})$ \\
$L_{7}^{(1)}$  & = & ${\lambda}{\cdot}(B^{2}+A^{2})$ \\
$L_{8}^{(1)}$  & = & ${\lambda}{\cdot}(B^{4}-A^{4})$ \\
$L_{9}^{(1)}$  & = & ${\lambda}{\cdot}(B^{3}A-BA^{3})$ \\
$L_{10}^{(1)}$ & = & ${\lambda}{\cdot}(B^{2}-A^{2})$ \\
\end{tabular}
\end{center}
\end{table}

In this representation, we see the following:

\begin{table}[!hp]
\begin{center}
\begin{tabular}{rcl}
$H_{0}$ & = & $L_{1}^{(1)}+\frac{1}{2}L_{0}^{(1)}$ \\
$H_{4}$ & = & $H_{0} + (0.25)L_{5}^{(1)} + L_{6}^{(1)}$ \nonumber \\
        &   & $+ (1.5)L_{7}^{(1)} + (1.5)L_{4}^{(1)}$ \nonumber \\
        &   & $+ (3)L_{3}^{(1)} + (0.75)L_{2}^{(1)}$
\end{tabular}
\end{center}
\end{table}

\newpage
This representation is complete for our purposes because it satisfies two conditions:
\label{secTwoConditions}

\begin{enumerate}
\item $H_{4}$ can be completely represented by terms in the algebra.
\item No two terms can be commuted to create a third non-trivial term not shown in the group.  (Remember, $\lambda^{2} = 0$).
\end{enumerate}

\subsection{Step 2: Construct a General Lie Group Element}

In principle, the Lie group element could be constructed from all terms in the Lie algebra, like so:

\begin{equation}
U = \textrm{exp}(\sum\limits_{k=0}^{10} \alpha_{k}{\cdot}L_{k})
\end{equation}

But, by nature of the Hammard lemma (see Section \ref{ssecTransformation}), we can choose to exclude all terms that commute with $H_{0}$.  So we construct $U$ as follows:

\begin{equation}
U = \textrm{exp}(\alpha_{5}L_{5}+\alpha_{6}L_{6}+\alpha_{7}L_{7}+\alpha_{8}L_{8}+\alpha_{9}L_{9}+\alpha_{10}L_{10})
\end{equation}

This gives us 6 constants we tune in order make this Lie group element a transformation of basis between perturbed and unperturbed eigenstates.

\subsection{Step 3: Use the Hammard Lemma to Compute our Lie Group Element}

It is our goal to choose a $U$ such that the following is true:

\begin{equation}
  H_{4} - U^{\dagger}H_{0}U = \Lambda_{4}
\label{eqDefinitionOfLambdaOperator}
\end{equation}
where $[U,\Lambda_{4}] = 0 + O(\lambda^{2})$

\subsubsection{Step 3.1: Expand $U^{\dagger}H_{0}U$ by the Hammard Lemma}
\label{ssecTransformation}

\begin{equation}
U^{\dagger}H_{0}U = H_{0} + [-X,H_{0}] + \frac{1}{2!}([-X,[-X,H_{0}]]) + {\cdots}
\end{equation}

where $X = \alpha_{5}L_{5}+\alpha_{6}L_{6}+\alpha_{7}L_{7}+\alpha_{8}L_{8}+\alpha_{9}L_{9}+\alpha_{10}L_{10}$.

To first order in $\lambda$ this simplifies to:

\begin{equation}
U^{\dagger}H_{0}U = H_{0} + [-X,H_{0}]
\label{eqHammardFirstOrder}
\end{equation}

Performing the commutator of Equation \ref{eqHammardFirstOrder} and normal ordering, we get the following:

\begin{eqnarray}
[-X,H_{0}] & = & [H_{0},X] \nonumber \\
           & = & [BA,X] \nonumber \\
           & = & {\lambda}{\cdot}[BA,\alpha_{5}(B^{4}+A^{4})+\alpha_{6}(B^{3}A+BA^{3}) \\
           &   &                     +\alpha_{7}(B^{2}+A^{2})+\alpha_{8}(B^{4}-A^{4}) \nonumber \\
           &   &                     +\alpha_{9}(B^{3}A-BA^{3})+\alpha_{10}(B^{2}-A^{2})] \nonumber \\
           &   & \nonumber \\
           & = & {\lambda}{\cdot}( (4\alpha_8){\cdot}(B^{4}+A^{4}) + (4\alpha_5){\cdot}(B^{4}-A^{4}) \\
           &   &  + (2\alpha_9){\cdot}(B^{3}A+BA^{3}) + (2\alpha_6){\cdot}(B^{3}A-BA^{3}) \nonumber \\
           &   &  + (2\alpha_{10}){\cdot}(B^{2}+A^{2}) + (2\alpha_7){\cdot}(B^{2}-A^{2})) \nonumber \\
           &   & \nonumber \\
           & = & (4\alpha_8){\cdot}L_{5} + (4\alpha_5){\cdot}L_{8} \\
           &   &  + (2\alpha_9){\cdot}L_{6} + (2\alpha_6){\cdot}L_{9} \nonumber \\
           &   &  + (2\alpha_{10}){\cdot}L_{7} + (2\alpha_7){\cdot}L_{10} \nonumber \\
           &   & \nonumber
\end{eqnarray}

\subsubsection{Step 3.2: Tune $\alpha_{k}$ so $\Lambda_{4}$ is a Number Operator}

From Equations \ref{eqDefinitionOfLambdaOperator} and \ref{eqHammardFirstOrder},

\begin{eqnarray}
\Lambda_{4} & = & H_{4} - U^{\dagger}H_{0}U \nonumber \\
            &   & \nonumber \\
            & = & (0.25-4\alpha_8)L_{5} + (1-2\alpha_9)L_{6} \\
            &   & + (1.5-2\alpha_{10})L_{7} + (1.5)L_{4} \nonumber \\
            &   & + (3)L_{3} + (0.75)L_{2} \nonumber \\
            &   & + (-4\alpha_5){\cdot}L_{8} + (-2\alpha_6){\cdot}L_{9} \nonumber \\
            &   & + (-2\alpha_7){\cdot}L_{10} \nonumber
\end{eqnarray}

Now, using our knowledge that $\Lambda_{4}$ must commute with $U$, we know that $\Lambda_{4}$ cannot have terms involving $L_{5}, L_{6}, L_{7}, L_{8}, L_{9}$ or $L_{10}$.  Thus, the alphas must be tuned such that:

\begin{table}[!h]
\begin{center}
\begin{tabular}{rclcrcl}
 $(-4\alpha_5)       $ & = & 0 & $\rightarrow$ & $\alpha_5   $ & = & $0           $  \\ 
                       &   &   &               &               &   &                 \\
 $(-2\alpha_6)       $ & = & 0 & $\rightarrow$ & $\alpha_6   $ & = & $0           $  \\
                       &   &   &               &               &   &                 \\
 $(-2\alpha_7)       $ & = & 0 & $\rightarrow$ & $\alpha_7   $ & = & $0           $  \\
                       &   &   &               &               &   &                 \\
 $(0.25-4\alpha_8)   $ & = & 0 & $\rightarrow$ & $\alpha_8   $ & = & $\frac{1}{16}$  \\
                       &   &   &               &               &   &                 \\
 $(1-2\alpha_9)      $ & = & 0 & $\rightarrow$ & $\alpha_9   $ & = & $\frac{1}{2} $  \\
                       &   &   &               &               &   &                 \\
 $(1.5-2\alpha_{10}) $ & = & 0 & $\rightarrow$ & $\alpha_{10}$ & = & $\frac{3}{4} $ 
\end{tabular}
\end{center}
\end{table}

\newpage

Which leaves:

\begin{eqnarray}
\Lambda_{4} & = & \frac{3}{2}L_{4} + 3L_{3} + \frac{3}{4}L_{2} \\
            & = & \frac{3}{2}\lambda(B^{2}A^{2}) + 3\lambda(BA) + \frac{3}{4}\lambda 
\end{eqnarray}

We have now completed the computation of the quartic oscillator to first order in $\lambda$ for all states.  Moving on...

\section{Solution to second order in $\lambda$}

\subsection{Step 1: Identify All Elements of the Lie Algebra}

We retain all of the previous terms in the Lie algebra, $L_{0}$ through $L_{10}$, but need to consider that $\lambda^{2} \neq 0$.  Thus, the commutators of all of these terms with one another are fair game.  The non-trivial commutators are shown in Table \ref{tabSecondOrderCommutators}.

\begin{table}[!hp]
\begin{center}
\begin{tabular}{rcl}
\hline
$[L1,L5]$ & = & ${\lambda}{\cdot}(4){\cdot}(B^{4}-A^{4})$ \\
\hline
$[L1,L6]$ & = & ${\lambda}{\cdot}(2){\cdot}(B^{3}A-BA^{3})$ \\
\hline
$[L1,L7]$ & = & ${\lambda}{\cdot}(2){\cdot}(B^{2}-A^{2})$ \\
\hline
$[L1,L8]$ & = & ${\lambda}{\cdot}(4){\cdot}(B^{4}+A^{4})$ \\
\hline
$[L1,L9]$ & = & ${\lambda}{\cdot}(2){\cdot}(B^{3}A+BA^{3})$ \\
\hline
$[L1,L10]$ & = & ${\lambda}{\cdot}(2){\cdot}(B^{2}+A^{2})$ \\
\hline
$[L3,L5]$ & = & ${\lambda}^2{\cdot}(4){\cdot}(B^{4}-A^{4})$ \\
\hline
$[L3,L6]$ & = & ${\lambda}^2{\cdot}(2){\cdot}(B^{3}A-BA^{3})$ \\
\hline
$[L3,L7]$ & = & ${\lambda}^2{\cdot}(2){\cdot}(B^{2}-A^{2})$ \\
\hline
$[L3,L8]$ & = & ${\lambda}^2{\cdot}(4){\cdot}(B^{4}+A^{4})$ \\
\hline
$[L3,L9]$ & = & ${\lambda}^2{\cdot}(2){\cdot}(B^{3}A+BA^{3})$ \\
\hline
$[L3,L10]$ & = & ${\lambda}^2{\cdot}(2){\cdot}(B^{2}+A^{2})$ \\
\hline
$[L4,L5]$ & = & ${\lambda}^2{\cdot}(8){\cdot}(B^{5}A-BA^{5}) + {\lambda}^2{\cdot}(12){\cdot}(B^{4}-A^{4})$ \\
\hline
$[L4,L6]$ & = & ${\lambda}^2{\cdot}(4){\cdot}(B^{4}A^{2}-B^{2}A^{4}) + {\lambda}^2{\cdot}(6){\cdot}(B^{3}A-BA^{3})$ \\
\hline
$[L4,L7]$ & = & ${\lambda}^2{\cdot}(4){\cdot}(B^{3}A-BA^{3}) + {\lambda}^2{\cdot}(2){\cdot}(B^{2}-A^{2})$ \\
\hline
$[L4,L8]$ & = & ${\lambda}^2{\cdot}(8){\cdot}(B^{5}A+BA^{5}) + {\lambda}^2{\cdot}(12){\cdot}(B^{4}+A^{4})$ \\
\hline
$[L4,L9]$ & = & ${\lambda}^2{\cdot}(4){\cdot}(B^{4}A^{2}+B^{2}A^{4}) + {\lambda}^2{\cdot}(6){\cdot}(B^{3}A+BA^{3})$ \\
\hline
$[L4,L10]$ & = & ${\lambda}^2{\cdot}(4){\cdot}(B^{3}A+BA^{3}) + {\lambda}^2{\cdot}(2){\cdot}(B^{2}+A^{2})$ \\
\hline
$[L5,L6]$ & = & ${\lambda}^2{\cdot}(-4){\cdot}(B^{6}-A^{6}) + {\lambda}^2{\cdot}(-12){\cdot}(B^{4}A^{2}-B^{2}A^{4})$ \\
          &   & $ + {\lambda}^2{\cdot}(-36){\cdot}(B^{3}A-BA^{3}) + {\lambda}^2{\cdot}(-24){\cdot}(B^{2}-A^{2})$ \\
\hline
$[L5,L7]$ & = & ${\lambda}^2{\cdot}(-8){\cdot}(B^{3}A-BA^{3}) + {\lambda}^2{\cdot}(-12){\cdot}(B^{2}-A^{2})$ \\
\hline
$[L5,L8]$ & = & ${\lambda}^2{\cdot}(32){\cdot}B^{3}A^{3} + {\lambda}^2{\cdot}(144){\cdot}B^{2}A^{2}$ \\
          &   & $ + {\lambda}^2{\cdot}(192){\cdot}BA + {\lambda}^2{\cdot}(48)$ \\
\hline
$[L5,L9]$ & = & ${\lambda}^2{\cdot}(-4){\cdot}(B^{6}+A^{6}) + {\lambda}^2{\cdot}(12){\cdot}(B^{4}A^{2}+B^{2}A^{4})$ \\
          &   & $ + {\lambda}^2{\cdot}(36){\cdot}(B^{3}A+BA^{3}) + {\lambda}^2{\cdot}(24){\cdot}(B^{2}+A^{2})$ \\
\hline
$[L5,L10]$ & = & ${\lambda}^2{\cdot}(8){\cdot}(B^{3}A+BA^{3}) + {\lambda}^2{\cdot}(12){\cdot}(B^{2}+A^{2})$ \\
\hline
$[L6,L7]$ & = & ${\lambda}^2{\cdot}(2){\cdot}(B^{4}-A^{4})$ \\
\hline
$[L6,L8]$ & = & ${\lambda}^2{\cdot}(4){\cdot}(B^{6}+A^{6}) + {\lambda}^2{\cdot}(12){\cdot}(B^{4}A^{2}+B^{2}A^{4})$ \\
          &   & $ + {\lambda}^2{\cdot}(36){\cdot}(B^{3}A+BA^{3}) + {\lambda}^2{\cdot}(24){\cdot}(B^{2}+A^{2})$ \\
\hline
$[L6,L9]$ & = & ${\lambda}^2{\cdot}(16){\cdot}B^{3}A^{3} + {\lambda}^2{\cdot}(36){\cdot}B^{2}A^{2}$ \\
          &   & $ + {\lambda}^2{\cdot}(12){\cdot}BA$ \\
\hline
$[L6,L10]$ & = & ${\lambda}^2{\cdot}(2){\cdot}(B^{4}+A^{4}) + {\lambda}^2{\cdot}(12){\cdot}B^{2}A^{2}$ \\
          &   & $ + {\lambda}^2{\cdot}(12){\cdot}BA$ \\
\hline
$[L7,L8]$ & = & ${\lambda}^2{\cdot}(8){\cdot}(B^{3}A+BA^{3}) + {\lambda}^2{\cdot}(12){\cdot}(B^{2}+A^{2})$ \\
\hline
$[L7,L9]$ & = & ${\lambda}^2{\cdot}(-2){\cdot}(B^{4}+A^{4}) + {\lambda}^2{\cdot}(12){\cdot}B^{2}A^{2}$ \\
          &   & $ + {\lambda}^2{\cdot}(12){\cdot}BA$ \\
\hline
$[L7,L10]$ & = & ${\lambda}^2{\cdot}(8){\cdot}BA + {\lambda}^2{\cdot}(4)$ \\
\hline
$[L8,L9]$ & = & ${\lambda}^2{\cdot}(-4){\cdot}(B^{6}-A^{6}) + {\lambda}^2{\cdot}(12){\cdot}(B^{4}A^{2}-B^{2}A^{4})$ \\
          &   & $ + {\lambda}^2{\cdot}(36){\cdot}(B^{3}A-BA^{3}) + {\lambda}^2{\cdot}(24){\cdot}(B^{2}-A^{2})$ \\
\hline
$[L8,L10]$ & = & ${\lambda}^2{\cdot}(8){\cdot}(B^{3}A-BA^{3}) + {\lambda}^2{\cdot}(12){\cdot}(B^{2}-A^{2})$ \\
\hline
$[L9,L10]$ & = & ${\lambda}^2{\cdot}(2){\cdot}(B^{4}-A^{4})$ \\
\hline
\end{tabular}
\caption{All non-trivial commutators of terms in the first order Lie Algebra -- \emph{i.e.} a computed list of terms associated with the second order Lie Algebra.\label{tabSecondOrderCommutators}}
\end{center}
\end{table}

By inspection, we can see that our first order Lie Algebra must be extended.  The complete list is shown in Table \ref{tabSecondOrderCompleteTermList}.

\begin{table}[!hp]
\begin{center}
\begin{tabular}{rcl}
               &   & \textbf{Nonperturbative Terms} ($\lambda^{k}$ where $k=0$) \\
\hline
$L_{0}^{(2)}$  & = & $I = 1$ \\
$L_{1}^{(2)}$  & = & $BA$ \\
               &   & \textbf{First Order Terms} ($\lambda^{k}$ where $k=1$) \\
\hline         
$L_{2}^{(2)}$  & = & ${\lambda}{\cdot}I = {\lambda}$ \\
$L_{3}^{(2)}$  & = & ${\lambda}{\cdot}BA$ \\
$L_{4}^{(2)}$  & = & ${\lambda}{\cdot}B^{2}A^{2}$ \\
               &   & \\
$L_{5}^{(2)}$  & = & ${\lambda}{\cdot}(B^{4}+A^{4})$ \\
$L_{6}^{(2)}$  & = & ${\lambda}{\cdot}(B^{3}A+BA^{3})$ \\
$L_{7}^{(2)}$  & = & ${\lambda}{\cdot}(B^{2}+A^{2})$ \\
$L_{8}^{(2)}$  & = & ${\lambda}{\cdot}(B^{4}-A^{4})$ \\
$L_{9}^{(2)}$  & = & ${\lambda}{\cdot}(B^{3}A-BA^{3})$ \\
$L_{10}^{(2)}$ & = & ${\lambda}{\cdot}(B^{2}-A^{2})$ \\
               &   & \textbf{Second Order Terms} ($\lambda^{k}$ where $k=2$) \\
\hline         
$L_{11}^{(2)}$ & = & ${\lambda}^{2}{\cdot}I = {\lambda}^{2}$ \\
$L_{12}^{(2)}$ & = & ${\lambda}^{2}{\cdot}BA$ \\
$L_{13}^{(2)}$ & = & ${\lambda}^{2}{\cdot}B^{2}A^{2}$ \\
$L_{14}^{(2)}$ & = & ${\lambda}^{2}{\cdot}B^{3}A^{3}$ \\
               &   & \\
$L_{15}^{(2)}$ & = & ${\lambda}^{2}{\cdot}(B^{6}+A^{6})$ \\
$L_{16}^{(2)}$ & = & ${\lambda}^{2}{\cdot}(B^{5}A+BA^{5})$ \\
$L_{17}^{(2)}$ & = & ${\lambda}^{2}{\cdot}(B^{4}A^{2}+B^{2}A^{4})$ \\
$L_{18}^{(2)}$ & = & ${\lambda}^{2}{\cdot}(B^{4}+A^{4})$ \\
$L_{19}^{(2)}$ & = & ${\lambda}^{2}{\cdot}(B^{3}A+BA^{3})$ \\
$L_{20}^{(2)}$ & = & ${\lambda}^{2}{\cdot}(B^{2}+A^{2})$ \\
$L_{21}^{(2)}$ & = & ${\lambda}^{2}{\cdot}(B^{6}-A^{6})$ \\
$L_{22}^{(2)}$ & = & ${\lambda}^{2}{\cdot}(B^{5}A-BA^{5})$ \\
$L_{23}^{(2)}$ & = & ${\lambda}^{2}{\cdot}(B^{4}A^{2}-B^{2}A^{4})$ \\
$L_{24}^{(2)}$ & = & ${\lambda}^{2}{\cdot}(B^{4}-A^{4})$ \\
$L_{25}^{(2)}$ & = & ${\lambda}^{2}{\cdot}(B^{3}A-BA^{3})$ \\
$L_{26}^{(2)}$ & = & ${\lambda}^{2}{\cdot}(B^{2}-A^{2})$
\end{tabular}
\caption{The list of all terms in the second order Lie algebra, derived (by inspection) from Table \ref{tabSecondOrderCommutators}. \label{tabSecondOrderCompleteTermList}}
\end{center}
\end{table}

The terms in Table \ref{tabSecondOrderCompleteTermList} form a complete representation, as it satisfies the two conditions from \ref{secTwoConditions}, where $\lambda^{2} \neq 0, \lambda^{3} = 0$.

\newpage

\subsection{Step 2: Construct a General Lie Group Element}

Once again, we may discard terms that commute with $H_{0}$, (the number operator terms, $L_{0}$ through $L_{4}$ and $L_{11}$ through $L_{14}$) when constructing the general Lie group element.

\begin{eqnarray}
U & = \textrm{exp}( & \beta_{5}L_{5}+\beta_{6}L_{6}+\beta_{7}L_{7}+\beta_{8}L_{8}+\beta_{9}L_{9}+\beta_{10}L_{10} \\
  &                 & +\beta_{15}L_{15}+\beta_{16}L_{16}+\beta_{17}L_{17}+\beta_{18}L_{18}+\beta_{19}L_{19} \nonumber \\
  &                 & +\beta_{20}L_{20}+\beta_{21}L_{21}+\beta_{22}L_{22}+\beta_{23}L_{23}+\beta_{24}L_{24} \nonumber \\
  &                 & +\beta_{25}L_{25}+\beta_{26}L_{26} ) \nonumber
\end{eqnarray}

\subsection{Step 3: Use the Hammard Lemma to Compute our Lie Group Element}

There are 18 elements to compute in this group element.  We proceed as before.

\subsubsection{Step 3.1: Expand $U^{\dagger}H_{0}U$ by the Hammard Lemma}

\begin{equation}
U^{\dagger}H_{0}U = H_{0} + [-X,H_{0}] + \frac{1}{2!}([-X,[-X,H_{0}]])
\end{equation}

where $X = \beta_{5}L_{5}+\beta_{6}L_{6}+\beta_{7}L_{7}+\beta_{8}L_{8}+\beta_{9}L_{9}+\beta_{10}L_{10}+\beta_{15}L_{15}+\beta_{16}L_{16}+\beta_{17}L_{17}+\beta_{18}L_{18}+\beta_{19}L_{19}+\beta_{20}L_{20}+\beta_{21}L_{21}+\beta_{22}L_{22}+\beta_{23}L_{23}+\beta_{24}L_{24}+\beta_{25}L_{25}+\beta_{26}L_{26}$.

\newpage

Taking the first term:

\begin{table}[!hp]
\begin{center}
\begin{tabular}{rcl}
$[-X,H_{0}]$ & = & ${\lambda}^2(6{\cdot}\beta_{21})(B^{6}+A^{6}) + {\lambda}^2(6{\cdot}\beta_{15})(B^{6}-A^{6})$ \\
 & & $ + {\lambda}^2(4{\cdot}\beta_{22})(B^{5}A+BA^{5}) + {\lambda}^2(4{\cdot}\beta_{16})(B^{5}A-BA^{5})$ \\
 & & $ + {\lambda}^2(2{\cdot}\beta_{23})(B^{4}A^{2}+B^{2}A^{4}) + {\lambda}^2(2{\cdot}\beta_{17})(B^{4}A^{2}-B^{2}A^{4})$ \\
 & & $ + {\lambda}(4{\cdot}\beta_{8})(B^{4}+A^{4}) + {\lambda}(4{\cdot}\beta_{5})(B^{4}-A^{4})$ \\
 & & $ + {\lambda}^2(4{\cdot}\beta_{24})(B^{4}+A^{4}) + {\lambda}^2(4{\cdot}\beta_{18})(B^{4}-A^{4})$ \\
 & & $ + {\lambda}(2{\cdot}\beta_{9})(B^{3}A+BA^{3}) + {\lambda}(2{\cdot}\beta_{6})(B^{3}A-BA^{3})$ \\
 & & $ + {\lambda}^2(2{\cdot}\beta_{25})(B^{3}A+BA^{3}) + {\lambda}^2(2{\cdot}\beta_{19})(B^{3}A-BA^{3})$ \\
 & & $ + {\lambda}(2{\cdot}\beta_{10})(B^{2}+A^{2}) + {\lambda}(2{\cdot}\beta_{7})(B^{2}-A^{2})$ \\
 & & $ + {\lambda}^2(2{\cdot}\beta_{26})(B^{2}+A^{2}) + {\lambda}^2(2{\cdot}\beta_{20})(B^{2}-A^{2})$
\end{tabular}
\end{center}
\end{table}

\newpage

And now, the second term:

$\frac{1}{2!}[-X,[-X,H_{0}]]$
\begin{table}[!hp]
\begin{center}
\begin{tabular}{rcl}
$\frac{1}{2!}[-X,[-X,H_{0}]]$ & = & ${\lambda}^2{\cdot}(-4{\cdot}{\beta}_{5}{\cdot}{\beta}_{6}-4{\cdot}{\beta}_{8}{\cdot}{\beta}_{9}-18{\cdot}{\beta}_{1}{\cdot}{\beta}_{15}){\cdot}(B^{6}+A^{6}) + {\lambda}^2{\cdot}(-4{\cdot}{\beta}_{5}{\cdot}{\beta}_{9}-4{\cdot}{\beta}_{6}{\cdot}{\beta}_{8}-18{\cdot}{\beta}_{1}{\cdot}{\beta}_{21}){\cdot}(B^{6}-A^{6})$ \\
 & & $ + {\lambda}^2{\cdot}(-8{\cdot}{\beta}_{1}{\cdot}{\beta}_{16}-16{\cdot}{\beta}_{4}{\cdot}{\beta}_{5}){\cdot}(B^{5}A+BA^{5}) + {\lambda}^2{\cdot}(-8{\cdot}{\beta}_{1}{\cdot}{\beta}_{22}-16{\cdot}{\beta}_{4}{\cdot}{\beta}_{8}){\cdot}(B^{5}A-BA^{5})$ \\
 & & $ + {\lambda}^2{\cdot}(-2{\cdot}{\beta}_{1}{\cdot}{\beta}_{17}-4{\cdot}{\beta}_{4}{\cdot}{\beta}_{6}-36{\cdot}{\beta}_{5}{\cdot}{\beta}_{6}+36{\cdot}{\beta}_{8}{\cdot}{\beta}_{9}){\cdot}(B^{4}A^{2}+B^{2}A^{4}) + {\lambda}^2{\cdot}(-2{\cdot}{\beta}_{1}{\cdot}{\beta}_{23}-4{\cdot}{\beta}_{4}{\cdot}{\beta}_{9}+36{\cdot}{\beta}_{5}{\cdot}{\beta}_{9}-36{\cdot}{\beta}_{6}{\cdot}{\beta}_{8}){\cdot}(B^{4}A^{2}-B^{2}A^{4})$ \\
 & & $ + {\lambda}{\cdot}(-8{\cdot}{\beta}_{1}{\cdot}{\beta}_{5}){\cdot}(B^{4}+A^{4}) + {\lambda}{\cdot}(-8{\cdot}{\beta}_{1}{\cdot}{\beta}_{8}){\cdot}(B^{4}-A^{4})$ \\
 & & $ + {\lambda}^2{\cdot}(-8{\cdot}{\beta}_{1}{\cdot}{\beta}_{18}-8{\cdot}{\beta}_{3}{\cdot}{\beta}_{5}-24{\cdot}{\beta}_{4}{\cdot}{\beta}_{5}){\cdot}(B^{4}+A^{4}) + {\lambda}^2{\cdot}(-8{\cdot}{\beta}_{1}{\cdot}{\beta}_{24}-8{\cdot}{\beta}_{3}{\cdot}{\beta}_{8}-24{\cdot}{\beta}_{4}{\cdot}{\beta}_{8}){\cdot}(B^{4}-A^{4})$ \\
 & & $ + {\lambda}{\cdot}(-2{\cdot}{\beta}_{1}{\cdot}{\beta}_{6}){\cdot}(B^{3}A+BA^{3}) + {\lambda}{\cdot}(-2{\cdot}{\beta}_{1}{\cdot}{\beta}_{9}){\cdot}(B^{3}A-BA^{3})$ \\
 & & $ + {\lambda}^2{\cdot}(-2{\cdot}{\beta}_{3}{\cdot}{\beta}_{6}-108{\cdot}{\beta}_{5}{\cdot}{\beta}_{6}+108{\cdot}{\beta}_{8}{\cdot}{\beta}_{9}-2{\cdot}{\beta}_{1}{\cdot}{\beta}_{19}-24{\cdot}{\beta}_{5}{\cdot}{\beta}_{7}+24{\cdot}{\beta}_{8}{\cdot}{\beta}_{10}-6{\cdot}{\beta}_{4}{\cdot}{\beta}_{6}-4{\cdot}{\beta}_{4}{\cdot}{\beta}_{7}){\cdot}(B^{3}A+BA^{3}) + {\lambda}^2{\cdot}(-2{\cdot}{\beta}_{3}{\cdot}{\beta}_{9}+108{\cdot}{\beta}_{5}{\cdot}{\beta}_{9}-108{\cdot}{\beta}_{6}{\cdot}{\beta}_{8}-2{\cdot}{\beta}_{1}{\cdot}{\beta}_{25}+24{\cdot}{\beta}_{5}{\cdot}{\beta}_{10}-24{\cdot}{\beta}_{7}{\cdot}{\beta}_{8}-6{\cdot}{\beta}_{4}{\cdot}{\beta}_{9}-4{\cdot}{\beta}_{4}{\cdot}{\beta}_{10}){\cdot}(B^{3}A-BA^{3})$ \\
 & & $ + {\lambda}^2{\cdot}(-72{\cdot}{\beta}_{5}{\cdot}{\beta}_{6}+72{\cdot}{\beta}_{8}{\cdot}{\beta}_{9}-36{\cdot}{\beta}_{5}{\cdot}{\beta}_{7}+36{\cdot}{\beta}_{8}{\cdot}{\beta}_{10}-2{\cdot}{\beta}_{1}{\cdot}{\beta}_{20}-2{\cdot}{\beta}_{3}{\cdot}{\beta}_{7}-2{\cdot}{\beta}_{4}{\cdot}{\beta}_{7}){\cdot}(B^{2}+A^{2}) + {\lambda}^2{\cdot}(72{\cdot}{\beta}_{5}{\cdot}{\beta}_{9}-72{\cdot}{\beta}_{6}{\cdot}{\beta}_{8}+36{\cdot}{\beta}_{5}{\cdot}{\beta}_{10}-36{\cdot}{\beta}_{7}{\cdot}{\beta}_{8}-2{\cdot}{\beta}_{1}{\cdot}{\beta}_{26}-2{\cdot}{\beta}_{3}{\cdot}{\beta}_{10}-2{\cdot}{\beta}_{4}{\cdot}{\beta}_{10}){\cdot}(B^{2}-A^{2})$ \\
 & & $ + {\lambda}{\cdot}(-2{\cdot}{\beta}_{1}{\cdot}{\beta}_{7}){\cdot}(B^{2}+A^{2}) + {\lambda}{\cdot}(-2{\cdot}{\beta}_{1}{\cdot}{\beta}_{10}){\cdot}(B^{2}-A^{2})$ \\
 & & $ + {\lambda}^2{\cdot}(-64{\cdot}{\beta}_{5}^{2}+64{\cdot}{\beta}_{8}^{2}-16{\cdot}{\beta}_{6}^{2}+16{\cdot}{\beta}_{9}^{2}){\cdot}B^{3}A^{3}$ \\
 & & $ + {\lambda}^2{\cdot}(-288{\cdot}{\beta}_{5}^{2}+288{\cdot}{\beta}_{8}^{2}-36{\cdot}{\beta}_{6}^{2}-24{\cdot}{\beta}_{6}{\cdot}{\beta}_{7}+36{\cdot}{\beta}_{9}^{2}+24{\cdot}{\beta}_{9}{\cdot}{\beta}_{10}){\cdot}B^{2}A^{2}$ \\
 & & $ + {\lambda}^2{\cdot}(-384{\cdot}{\beta}_{5}^{2}+384{\cdot}{\beta}_{8}^{2}-12{\cdot}{\beta}_{6}^{2}-24{\cdot}{\beta}_{6}{\cdot}{\beta}_{7}+12{\cdot}{\beta}_{9}^{2}+24{\cdot}{\beta}_{9}{\cdot}{\beta}_{10}-8{\cdot}{\beta}_{7}^{2}+8{\cdot}{\beta}_{10}^{2}){\cdot}BA$ \\
 & & $ + {\lambda}^2{\cdot}(-96{\cdot}{\beta}_{5}^{2}+96{\cdot}{\beta}_{8}^{2}-4{\cdot}{\beta}_{7}^{2}+4{\cdot}{\beta}_{10}^{2})$ \\
\end{tabular}
\end{center}
\end{table}

\newpage

And finally, the $\Lambda_{4}$

\begin{table}[!hp]
\begin{center}
\begin{tabular}{rcl}
$H_{4} - U^{\dagger}H_{0}U$ & = & $\Lambda_{4}$ \\
                            &   & \\
                            & = & $\frac{\lambda}{4}(A+B)^{4} - \left([-X,H_{0}] + \frac{1}{2!}[-X,[-X,H_{0}]]\right)$ \\
                            &   & \\
                            & = & ${\lambda}^2{\cdot}(-6{\cdot}{\beta}_{21}+4{\cdot}{\beta}_{5}{\cdot}{\beta}_{6}+4{\cdot}{\beta}_{8}{\cdot}{\beta}_{9}+18{\cdot}{\beta}_{1}{\cdot}{\beta}_{15}){\cdot}(B^{6}+A^{6})$ \\
                            &   & $ + {\lambda}^2{\cdot}(-6{\cdot}{\beta}_{15}+4{\cdot}{\beta}_{5}{\cdot}{\beta}_{9}+4{\cdot}{\beta}_{6}{\cdot}{\beta}_{8}+18{\cdot}{\beta}_{1}{\cdot}{\beta}_{21}){\cdot}(B^{6}-A^{6})$ \\
                            &   & $ + {\lambda}^2{\cdot}(-4{\cdot}{\beta}_{22}+8{\cdot}{\beta}_{1}{\cdot}{\beta}_{16}+16{\cdot}{\beta}_{4}{\cdot}{\beta}_{5}){\cdot}(B^{5}A+BA^{5})$ \\
                            &   & $ + {\lambda}^2{\cdot}(-4{\cdot}{\beta}_{16}+8{\cdot}{\beta}_{1}{\cdot}{\beta}_{22}+16{\cdot}{\beta}_{4}{\cdot}{\beta}_{8}){\cdot}(B^{5}A-BA^{5})$ \\
                            &   & $ + {\lambda}^2{\cdot}(-2{\cdot}{\beta}_{23}+2{\cdot}{\beta}_{1}{\cdot}{\beta}_{17}+4{\cdot}{\beta}_{4}{\cdot}{\beta}_{6}+36{\cdot}{\beta}_{5}{\cdot}{\beta}_{6}-36{\cdot}{\beta}_{8}{\cdot}{\beta}_{9}){\cdot}(B^{4}A^{2}+B^{2}A^{4})$ \\
                            &   & $ + {\lambda}^2{\cdot}(-2{\cdot}{\beta}_{17}+2{\cdot}{\beta}_{1}{\cdot}{\beta}_{23}+4{\cdot}{\beta}_{4}{\cdot}{\beta}_{9}-36{\cdot}{\beta}_{5}{\cdot}{\beta}_{9}+36{\cdot}{\beta}_{6}{\cdot}{\beta}_{8}){\cdot}(B^{4}A^{2}-B^{2}A^{4})$ \\
                            &   & $ + {\lambda}{\cdot}(0.25-4{\cdot}{\beta}_{8}+8{\cdot}{\beta}_{1}{\cdot}{\beta}_{5}){\cdot}(B^{4}+A^{4})$ \\
                            &   & $ + {\lambda}{\cdot}(-4{\cdot}{\beta}_{5}+8{\cdot}{\beta}_{1}{\cdot}{\beta}_{8}){\cdot}(B^{4}-A^{4})$ \\
                            &   & $ + {\lambda}^2{\cdot}(-4{\cdot}{\beta}_{24}+8{\cdot}{\beta}_{1}{\cdot}{\beta}_{18}+8{\cdot}{\beta}_{3}{\cdot}{\beta}_{5}+24{\cdot}{\beta}_{4}{\cdot}{\beta}_{5}){\cdot}(B^{4}+A^{4})$ \\
                            &   & $ + {\lambda}^2{\cdot}(-4{\cdot}{\beta}_{18}+8{\cdot}{\beta}_{1}{\cdot}{\beta}_{24}+8{\cdot}{\beta}_{3}{\cdot}{\beta}_{8}+24{\cdot}{\beta}_{4}{\cdot}{\beta}_{8}){\cdot}(B^{4}-A^{4})$ \\
                            &   & $ + {\lambda}{\cdot}(-2{\cdot}{\beta}_{9}+2{\cdot}{\beta}_{1}{\cdot}{\beta}_{6}){\cdot}(B^{3}A+BA^{3})$ \\
                            &   & $ + {\lambda}{\cdot}(-2{\cdot}{\beta}_{6}+2{\cdot}{\beta}_{1}{\cdot}{\beta}_{9}){\cdot}(B^{3}A-BA^{3})$ \\
                            &   & $ + {\lambda}^2{\cdot}(-2{\cdot}{\beta}_{25}+2{\cdot}{\beta}_{3}{\cdot}{\beta}_{6}+108{\cdot}{\beta}_{5}{\cdot}{\beta}_{6}-108{\cdot}{\beta}_{8}{\cdot}{\beta}_{9}+2{\cdot}{\beta}_{1}{\cdot}{\beta}_{19}+24{\cdot}{\beta}_{5}{\cdot}{\beta}_{7}-24{\cdot}{\beta}_{8}{\cdot}{\beta}_{10}+6{\cdot}{\beta}_{4}{\cdot}{\beta}_{6}+4{\cdot}{\beta}_{4}{\cdot}{\beta}_{7}){\cdot}(B^{3}A+BA^{3})$ \\
                            &   & $ + {\lambda}^2{\cdot}(-2{\cdot}{\beta}_{19}+2{\cdot}{\beta}_{3}{\cdot}{\beta}_{9}-108{\cdot}{\beta}_{5}{\cdot}{\beta}_{9}+108{\cdot}{\beta}_{6}{\cdot}{\beta}_{8}+2{\cdot}{\beta}_{1}{\cdot}{\beta}_{25}-24{\cdot}{\beta}_{5}{\cdot}{\beta}_{10}+24{\cdot}{\beta}_{7}{\cdot}{\beta}_{8}+6{\cdot}{\beta}_{4}{\cdot}{\beta}_{9}+4{\cdot}{\beta}_{4}{\cdot}{\beta}_{10}){\cdot}(B^{3}A-BA^{3})$ \\
                            &   & $ + {\lambda}{\cdot}(1.5-2{\cdot}{\beta}_{10}+2{\cdot}{\beta}_{1}{\cdot}{\beta}_{7}){\cdot}(B^{2}+A^{2})$ \\
                            &   & $ + {\lambda}{\cdot}(-2{\cdot}{\beta}_{7}+2{\cdot}{\beta}_{1}{\cdot}{\beta}_{10}){\cdot}(B^{2}-A^{2})$ \\
                            &   & $ + {\lambda}^2{\cdot}(-2{\cdot}{\beta}_{26}+72{\cdot}{\beta}_{5}{\cdot}{\beta}_{6}-72{\cdot}{\beta}_{8}{\cdot}{\beta}_{9}+36{\cdot}{\beta}_{5}{\cdot}{\beta}_{7}-36{\cdot}{\beta}_{8}{\cdot}{\beta}_{10}+2{\cdot}{\beta}_{1}{\cdot}{\beta}_{20}+2{\cdot}{\beta}_{3}{\cdot}{\beta}_{7}+2{\cdot}{\beta}_{4}{\cdot}{\beta}_{7}){\cdot}(B^{2}+A^{2})$ \\
                            &   & $ + {\lambda}^2{\cdot}(-2{\cdot}{\beta}_{20}-72{\cdot}{\beta}_{5}{\cdot}{\beta}_{9}+72{\cdot}{\beta}_{6}{\cdot}{\beta}_{8}-36{\cdot}{\beta}_{5}{\cdot}{\beta}_{10}+36{\cdot}{\beta}_{7}{\cdot}{\beta}_{8}+2{\cdot}{\beta}_{1}{\cdot}{\beta}_{26}+2{\cdot}{\beta}_{3}{\cdot}{\beta}_{10}+2{\cdot}{\beta}_{4}{\cdot}{\beta}_{10}){\cdot}(B^{2}-A^{2})$ \\
                            &   & $ + {\lambda}^2{\cdot}(64{\cdot}{\beta}_{5}^{2}-64{\cdot}{\beta}_{8}^{2}+16{\cdot}{\beta}_{6}^{2}-16{\cdot}{\beta}_{9}^{2}){\cdot}B^{3}A^{3}$ \\
                            &   & $ + {\lambda}{\cdot}(1.5){\cdot}B^{2}A^{2}$ \\
                            &   & $ + {\lambda}^2{\cdot}(288{\cdot}{\beta}_{5}^{2}-288{\cdot}{\beta}_{8}^{2}+36{\cdot}{\beta}_{6}^{2}+24{\cdot}{\beta}_{6}{\cdot}{\beta}_{7}-36{\cdot}{\beta}_{9}^{2}-24{\cdot}{\beta}_{9}{\cdot}{\beta}_{10}){\cdot}B^{2}A^{2}$ \\
                            &   & $ + {\lambda}{\cdot}(3){\cdot}BA$ \\
                            &   & $ + {\lambda}^2{\cdot}(384{\cdot}{\beta}_{5}^{2}-384{\cdot}{\beta}_{8}^{2}+12{\cdot}{\beta}_{6}^{2}+24{\cdot}{\beta}_{6}{\cdot}{\beta}_{7}-12{\cdot}{\beta}_{9}^{2}-24{\cdot}{\beta}_{9}{\cdot}{\beta}_{10}+8{\cdot}{\beta}_{7}^{2}-8{\cdot}{\beta}_{10}^{2}){\cdot}BA$ \\
                            &   & $ + {\lambda}{\cdot}(0.75)$ \\
                            &   & $ + {\lambda}^2{\cdot}(96{\cdot}{\beta}_{5}^{2}-96{\cdot}{\beta}_{8}^{2}+4{\cdot}{\beta}_{7}^{2}-4{\cdot}{\beta}_{10}^{2})$ \\
\end{tabular}
\end{center}
\end{table}


\subsubsection{Step 3.2: Tune $\beta_{k}$ so $\Lambda_{4}$ is a Number Operator}

Tuning the $\beta$ values is easily done by setting each term that is not a number operator in $\Lambda_{4}$ equal to zero, and solving for $\beta_{k}$.  The results are shown in Table \ref{tabSolvedBetaValues}.

\begin{table}[!hp]
\begin{center}
\begin{tabular}{rclrcl}
 $\beta_{5}$  & = & $ 0            $        & $\beta_{18}$ & = & $ 0              $ \\
 $\beta_{6}$  & = & $ 0            $        & $\beta_{19}$ & = & $ 0              $ \\
 $\beta_{7}$  & = & $ 0            $        & $\beta_{20}$ & = & $ 0              $ \\
 $\beta_{8}$  & = & $ \frac{1}{16} $        & $\beta_{21}$ & = & $ \frac{1}{48}   $ \\
 $\beta_{9}$  & = & $ \frac{1}{2}  $        & $\beta_{22}$ & = & $ 0              $ \\
 $\beta_{10}$ & = & $ \frac{3}{4}  $        & $\beta_{23}$ & = & $ -\frac{9}{16}  $ \\
 $\beta_{15}$ & = & $ 0            $        & $\beta_{24}$ & = & $ 0              $ \\
 $\beta_{16}$ & = & $ 0            $        & $\beta_{25}$ & = & $ -\frac{9}{4}   $ \\
 $\beta_{17}$ & = & $ 0            $        & $\beta_{26}$ & = & $ -\frac{63}{32} $ \\
\end{tabular}
\caption{Values for $\beta_{k}$. \label{tabSolvedBetaValues}}
\end{center}
\end{table}

The resulting form for $\Lambda_{4}$ is

\begin{equation}
\Lambda_{4} = \lambda(\frac{3}{2}B^{2}A^{2}+3BA+\frac{3}{4})+\lambda^{2}(\frac{-17}{4}B^{3}A^{3}+\frac{-153}{8}B^{2}A^{2}+(-18)BA+\frac{-21}{8})
\end{equation}

Not surprisingly, setting $\lambda^{2}$ equal to zero will return the first order perturbation result.

\end{document}
