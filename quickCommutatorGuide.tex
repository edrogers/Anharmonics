\documentclass{article}
\title{Quick and Dirty Guide to Using this Code}
\author{Ed Rogers}
\date{March 2016}
\begin{document}
   \maketitle

\begin{abstract}
This article is meant to give a brief explanation of how to type terms of a commutator into \texttt{main.c} and end up with a PDF \& \LaTeX\ formatted output of the resulting arithmetic.
\end{abstract}

\section{Getting the most recent version of the code}
Yada

\section{Test run as is}
Yada

\section{Explanation of what's in \texttt{Anharmonic.pdf}}
Yada

If we enter $X$, and ask it to print without grouping terms (see line commented out at Line 44), we just get what we put in, with some \LaTeX\ formatting (as seen on Page 1 of the output): 
\begin{table}[!hp]
\begin{center}
\begin{tabular}{rcl}
$X$ & = & ${\lambda}{\cdot}(-1{\cdot}{\alpha}_{17}){\cdot}ABA$ \\
 & & $ + {\lambda}{\cdot}(-12{\cdot}{\alpha}_{14}{\cdot}{\alpha}_{123}){\cdot}B^{2}ABA$ \\
\end{tabular}
\end{center}
\end{table}

If, however, we ask it to group terms, (see Line 85), terms are

\newpage

\begin{table}[!hp]
\begin{center}
\begin{tabular}{rcl}
$X$ & = & ${\lambda}{\cdot}(5{\cdot}{\alpha}_{1}){\cdot}(B^{4}+A^{4}) + {\lambda}{\cdot}(12{\cdot}{\alpha}_{1}){\cdot}(B^{4}-A^{4})$ \\
 & & $ + {\lambda}{\cdot}(1{\cdot}{\alpha}_{2}){\cdot}(B^{3}A-BA^{3})$ \\
 & & $ + {\lambda}{\cdot}(-3.5{\cdot}{\alpha}_{14}{\cdot}{\alpha}_{123}-0.5{\cdot}{\alpha}_{17}){\cdot}(B^{2}A+BA^{2}) + {\lambda}{\cdot}(-3.5{\cdot}{\alpha}_{14}{\cdot}{\alpha}_{123}+0.5{\cdot}{\alpha}_{17}){\cdot}(B^{2}A-BA^{2})$ \\
 & & $ + {\lambda}{\cdot}(1{\cdot}{\alpha}_{3}){\cdot}(B^{2}-A^{2})$ \\
 & & $ + {\lambda}{\cdot}(-7{\cdot}{\alpha}_{14}{\cdot}{\alpha}_{123}){\cdot}B^{3}A^{2}$ \\
 & & $ + {\lambda}{\cdot}(-1{\cdot}{\alpha}_{17}){\cdot}A$ \\
\end{tabular}
\end{center}
\end{table}

\newpage

\begin{table}[!hp]
\begin{center}
\begin{tabular}{rcl}
$Y$ & = & ${\lambda}{\cdot}(-3.5{\cdot}{\alpha}_{14}{\cdot}{\alpha}_{123}-0.5{\cdot}{\alpha}_{17}){\cdot}(B^{2}A+BA^{2}) + {\lambda}{\cdot}(-3.5{\cdot}{\alpha}_{14}{\cdot}{\alpha}_{123}+0.5{\cdot}{\alpha}_{17}){\cdot}(B^{2}A-BA^{2})$ \\
 & & $ + {\lambda}{\cdot}(1{\cdot}{\alpha}_{3}){\cdot}(B^{2}-A^{2})$ \\
 & & $ + {\lambda}{\cdot}(-7{\cdot}{\alpha}_{14}{\cdot}{\alpha}_{123}){\cdot}B^{3}A^{2}$ \\
 & & $ + {\lambda}{\cdot}(1{\cdot}{\alpha}_{1}){\cdot}B^{4}$ \\
 & & $ + {\lambda}{\cdot}(1{\cdot}{\alpha}_{2}){\cdot}B^{3}A$ \\
 & & $ + {\lambda}{\cdot}(-1{\cdot}{\alpha}_{17}){\cdot}A$ \\
\end{tabular}
\end{center}
\end{table}

\newpage

\begin{table}[!hp]
\begin{center}
\begin{tabular}{rcl}
$Z$ & = & ${\lambda}{\cdot}(-1{\cdot}{\alpha}_{17}){\cdot}ABA$ \\
 & & $ + {\lambda}{\cdot}(-12{\cdot}{\alpha}_{14}{\cdot}{\alpha}_{123}){\cdot}B^{2}ABA$ \\
\end{tabular}
\end{center}
\end{table}

\newpage

\begin{table}[!hp]
\begin{center}
\begin{tabular}{rcl}
$[-X,H_{0}]$ & = & ${\lambda}{\cdot}(48{\cdot}{\alpha}_{1}){\cdot}(B^{4}+A^{4}) + {\lambda}{\cdot}(20{\cdot}{\alpha}_{1}){\cdot}(B^{4}-A^{4})$ \\
 & & $ + {\lambda}{\cdot}(2{\cdot}{\alpha}_{2}){\cdot}(B^{3}A+BA^{3})$ \\
 & & $ + {\lambda}{\cdot}(-3.5{\cdot}{\alpha}_{14}{\cdot}{\alpha}_{123}+0.5{\cdot}{\alpha}_{17}){\cdot}(B^{2}A+BA^{2}) + {\lambda}{\cdot}(-3.5{\cdot}{\alpha}_{14}{\cdot}{\alpha}_{123}-0.5{\cdot}{\alpha}_{17}){\cdot}(B^{2}A-BA^{2})$ \\
 & & $ + {\lambda}{\cdot}(2{\cdot}{\alpha}_{3}){\cdot}(B^{2}+A^{2})$ \\
 & & $ + {\lambda}{\cdot}(-7{\cdot}{\alpha}_{14}{\cdot}{\alpha}_{123}){\cdot}B^{3}A^{2}$ \\
 & & $ + {\lambda}{\cdot}(1{\cdot}{\alpha}_{17}){\cdot}A$ \\
\end{tabular}
\end{center}
\end{table}

\end{document}
