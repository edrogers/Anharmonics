\documentclass{article}
\title{Quick and Dirty Guide to Using this Code}
\author{Ed Rogers}
\date{March 2016}
\begin{document}
   \maketitle

\begin{abstract}
This article is meant to give a brief explanation of how to type terms of a commutator into \texttt{main.c} and end up with a PDF \& \LaTeX\ formatted output of the resulting arithmetic.
\end{abstract}

\section{Getting the most recent version of the code}

First, set up a new directory, and download the latest copy of the code into it. You can visit:
\newline
\newline
\texttt{https://github.com/edrogers/Anharmonics}
\newline
\newline
The simplest way is to hit ``Download ZIP'' in the upper right, and just unzip the downloaded folder to your new directory. Use the Unix \texttt{unzip} command, and you should be ready to go.

\section{Test run as is}

Now, to get started, try compiling the code, running it, running the scripts that convert from \LaTeX\ to PDF, and checking the PDF output.

\subsection{Compile the code}

Type the following: 
\newline
\newline
\texttt{make}
\newline
\newline

\subsection{Run the code (producing a .tex file -- \texttt{Anharmonic.tex})}

Type the following: 
\newline
\newline
\texttt{./main}
\newline
\newline

\subsection{Convert \texttt{Anharmonic.tex} to \texttt{Anharmonic.pdf}}

Type the following: 
\newline
\newline
\texttt{make tex}
\newline
\newline

Now you should be ready to inspect the code output at \texttt{Anharmonic.tex} \& \texttt{Anharmonic.pdf}

\section{Explanation of what's in \texttt{Anharmonic.pdf}}

This section is where we roll up our sleeves and start looking at how \texttt{main.c} ends up producing \texttt{Anharmonic.tex}.

\subsection{Entering a random polynomial into \texttt{main.c}}

To start with, we enter a random polynomial of operators into our code, just to show the syntax of input. We start with $X$, such that:

\begin{table}[!hp]
\begin{center}
\begin{tabular}{rcl}
$X$ & = & ${\lambda}{\cdot}(-4{\cdot}{\alpha}_{17}){\cdot}ABA$ \\
 & & $ + {\lambda}{\cdot}(-12{\cdot}{\alpha}_{14}{\cdot}{\alpha}_{123}){\cdot}B^{2}ABA$ \\
\end{tabular}
\end{center}
\end{table}

In \texttt{main.c}, this $X$ is entered using lines 33-36. Lines 38-62 then print $X$ as a \LaTeX\ table to the file \texttt{Anharmonic.tex}. As you can see looking at \texttt{Anharmonic.pdf}, this $X$ appears on the first page.

\subsection{Using \texttt{main.c} to rearrange terms in a polynomial to form elements of $L^n_m$}

Obviously, the above $X$ is quite artificial, but as a demonstration of the use of \texttt{main.c}, let's un-comment Line 44, and re-run the code. Remember, the re-run procedure is:
\begin{itemize}
\item \texttt{make}
\item \texttt{./main}
\item \texttt{make tex}
\end{itemize}

The result in \texttt{Anharmonic.tex} (and \texttt{Anharmonic.pdf}) will be:

\begin{table}[h!]
\begin{center}
\begin{tabular}{rcl}
$X$ & = & ${\lambda}{\cdot}(-6{\cdot}{\alpha}_{14}{\cdot}{\alpha}_{123}-2{\cdot}{\alpha}_{17}){\cdot}(B^{2}A+BA^{2}) + {\lambda}{\cdot}(-6{\cdot}{\alpha}_{14}{\cdot}{\alpha}_{123}+2{\cdot}{\alpha}_{17}){\cdot}(B^{2}A-BA^{2})$ \\
 & & $ + {\lambda}{\cdot}(-12{\cdot}{\alpha}_{14}{\cdot}{\alpha}_{123}){\cdot}B^{3}A^{2}$ \\
 & & $ + {\lambda}{\cdot}(-4{\cdot}{\alpha}_{17}){\cdot}A$ \\
\end{tabular}
\end{center}
\end{table}
\newpage
So, calling \texttt{x.groupTerms();} has done a lot of the grunt-work of performing commutators and creating group elements from our original $X$.

\subsection{Performing a commutator}

Let's look at the \texttt{Anharmonic Notes (1).pdf} that you sent me, dated February 24, 2016, via email on Feb 25th. On Page 10, about halfway down, you show an operator $X$, (which I'll call $Y$, for clarity here), given by:
\newline
\newline
$Y$ = $\lambda(\alpha_{20}L_0^2 + \alpha_{31}L_1^3 + \alpha_{40}L_0^4) + \lambda^2(\beta_{20}L_0^2 + \beta_{31}L_1^3 + \beta_{40}L_0^4 +\beta_{42}L_2^4 + \beta_{60}L_0^6)$
\newline
\newline
This is entered into \texttt{main.c} on lines 74-91. The result is printed into \LaTeX\ format in lines 93-117, and appears as follows: 

\begin{table}[!hp]
\begin{center}
\begin{tabular}{rcl}
$Y$ & = & ${\lambda}^2{\cdot}(1{\cdot}{\alpha}_{10060}){\cdot}(B^{6}-A^{6})$ \\
 & & $ + {\lambda}^2{\cdot}(1{\cdot}{\alpha}_{10042}){\cdot}(B^{4}A^{2}-B^{2}A^{4})$ \\
 & & $ + {\lambda}{\cdot}(1{\cdot}{\alpha}_{40}){\cdot}(B^{4}+A^{4})$ \\
 & & $ + {\lambda}^2{\cdot}(1{\cdot}{\alpha}_{10040}){\cdot}(B^{4}-A^{4})$ \\
 & & $ + {\lambda}{\cdot}(1{\cdot}{\alpha}_{31}){\cdot}(B^{3}A-BA^{3})$ \\
 & & $ + {\lambda}^2{\cdot}(1{\cdot}{\alpha}_{10031}){\cdot}(B^{3}A-BA^{3})$ \\
 & & $ + {\lambda}{\cdot}(1{\cdot}{\alpha}_{20}){\cdot}(B^{2}-A^{2})$ \\
 & & $ + {\lambda}^2{\cdot}(1{\cdot}{\alpha}_{10020}){\cdot}(B^{2}-A^{2})$ \\
\end{tabular}
\end{center}
\end{table}

(Note: Where your document has used $\beta_{40}$, I have substituted $\alpha_{10040}$. i.e., $\beta_{x} \rightarrow \alpha_{10000+x}$.)

Lines 121-149 define and print $H_{0}$. Then Line 154 performs the commutator, $[Y,H_0]$, which is then printed by Lines 156-180.

\newpage

\subsection{Performing multiple commutators}

Looking at the \texttt{Anharmonic Notes (1).pdf}, on page 16, it looks like you have a lot of double commutators to perform. Let's have \texttt{main.c} do one. ``Commutator 2.3.3'' is $[L^2_0,[L^3_1,\bar{L}^3_1]]$. These three elements are defined across Lines 192-285. Then they are commuted pretty quickly using the line:
\newline
\newline
\texttt{Polynomial Commutator\_2\_3\_3 = L20.commutedWith(L31.commutedWith(L31Bar));}
\newline
\newline
The result is printed in the same way as usual:
\begin{table}[!hp]
\begin{center}
\begin{tabular}{rcl}
$[L^{2}_{0},[L^{3}_{1},\bar{L}^{3}_{1}]]$ & = & $(96){\cdot}(B^{4}A^{2}+B^{2}A^{4})$ \\
 & & $ + (240){\cdot}(B^{3}A+BA^{3})$ \\
 & & $ + (96){\cdot}(B^{2}+A^{2})$ \\
\end{tabular}
\end{center}
\end{table}

Okay, that should get you started!
 

\end{document}
